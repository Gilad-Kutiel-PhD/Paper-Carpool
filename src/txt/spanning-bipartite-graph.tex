Given a directed graph $G = (V, E)$,
a weight function $w : E \rightarrow \R$, 
and a spanning tree $T$ of the underlying graph, 
we refer to the directed bipartite
\footnote{a graph whose underlying graph is a tree, must be a bipartite graph}
graph resulting from directing the edges 
of $T$ according to their original direction
\footnote{if there are antisymmetric edges in $G$, 
choose an arbitrary direction to the underlying edge}
as the \emph{spanning bipartite graph} of $G$.

For a given spanning bipartite graph $G = (V = (L, R), E)$, 
denote by $E_{LR}$ the edges directed from a vertex in $L$ to a vertex in $R$,
and by $E_{RL}$ the edges directed from a vertex in $R$ to a vertex in $L$,
For a set of edges $E$ let $w(E) = \sum_{e \in E}{w(e)}$, 
then, clearly, the following property holds:

\begin{observation}
\label{ob:geq_half}
$ \max\{w(E_{LR}), w(E_{RL})\}  \geq \frac{w(E)}{2} $
\end{observation}

An example spanning bipartite graph is illustrated in 
Figure~\ref{fig:spanning-bipartite-graph}

\begin{figure}
\centering
\subfigure[]{
\label{sub:graph}
\begin{tikzpicture}[every node/.style={default node}]
\node(1) at (0,0) {1};
\node(2) at (2,0) {2};
\node(3) at (0,2) {3};
\node(4) at (2,2) {4};
\node(5) at (1,3.5) {5};

\draw[->, very thick](1) -- (2);
\draw[->, very thick](1) -- (3);
\draw[->, very thick](4) -- (1);
\draw[->, very thick](2) -- (3);
\draw[->, very thick](2) -- (4);
\draw[->, very thick](3) -- (4);
\draw[->, very thick](3) -- (5);
\draw[->, very thick](5) -- (4);

\end{tikzpicture}
}
\hfill
%
\subfigure[]{
\label{sub:tree}
\begin{tikzpicture}[every node/.style={default node}]
\node(1) at (0,0) {1};
\node(2) at (2,0) {2};
\node(3) at (0,2) {3};
\node(4) at (2,2) {4};
\node(5) at (1,3.5) {5};

\draw[very thick](1) -- (2);
\draw[very thick](1) -- (4);
\draw[very thick](3) -- (4);
\draw[very thick](5) -- (4);

\end{tikzpicture}
}
\hfill
%
\subfigure[]{
\label{sub:bipartite}
\begin{tikzpicture}[every node/.style={default node}]
\node(1) at (0,0) {1};
\node(2) at (2,0) {2};
\node(3) at (0,2) {3};
\node(4) at (2,2) {4};
\node(5) at (0,3.5) {5};

\draw[->, very thick, dashed, blue](1) -- (2);
\draw[->, very thick, dotted, red](4) -- (1);
\draw[->, very thick, dashed, blue](3) -- (4);
\draw[->, very thick, dashed, blue](5) -- (4);

\end{tikzpicture}
}
\hfill
\caption{
\label{fig:spanning-bipartite-graph}
\subref{sub:graph} is a directed graph, 
\subref{sub:tree} is a spanning tree, and  
\subref{sub:bipartite} is a spanning bipartite graph.
the blue, dashed edges form $E_{LR}$, 
and the red, dotted edge is $E_{RL}$  
}
\end{figure}