\label{sub:cm}
We now present a 3-app\-roximation algorithm for the \textsc{\CARPOOL{}} problem.
This algorithm acts in two phases.
In the first phase it computes a subgraph of $G$ 
where $d_{in}(v) \leq c(v)$, and $d_{out}(v) \leq 1$ for every $v \in V$, 
then, in the second phase it turns this subgraph to a feasible matching
by removing some of the edges.

Given a directed graph $G = (V, E)$,
a weight function $w : E \rightarrow \R$, 
and a spanning tree $T$ of the underlying graph, 
we refer to the directed bipartite~\footnote{
a graph whose underlying graph is a tree, must be a bipartite graph}
graph resulting from directing the edges 
of $T$ according to their original direction~\footnote{
if there are antisymmetric edges in $G$, 
choose an arbitrary direction to the underlying edge}
as the \emph{spanning bipartite graph} of $G$.
An example of a spanning bipartite graph is illustrated in 
Figure~\ref{fig:spanning-bipartite-graph}

\begin{figure}
\centering
\subfigure[]{
\label{sub:graph}
\begin{tikzpicture}[every node/.style={default node}]
\node(1) at (0,0) {1};
\node(2) at (2,0) {2};
\node(3) at (0,2) {3};
\node(4) at (2,2) {4};
\node(5) at (1,3.5) {5};

\draw[->, very thick](1) -- (2);
\draw[->, very thick](1) -- (3);
\draw[->, very thick](4) -- (1);
\draw[->, very thick](2) -- (3);
\draw[->, very thick](2) -- (4);
\draw[->, very thick](3) -- (4);
\draw[->, very thick](3) -- (5);
\draw[->, very thick](5) -- (4);

\end{tikzpicture}
}
\hfill
%
\subfigure[]{
\label{sub:tree}
\begin{tikzpicture}[every node/.style={default node}]
\node(1) at (0,0) {1};
\node(2) at (2,0) {2};
\node(3) at (0,2) {3};
\node(4) at (2,2) {4};
\node(5) at (1,3.5) {5};

\draw[very thick](1) -- (2);
\draw[very thick](1) -- (4);
\draw[very thick](3) -- (4);
\draw[very thick](5) -- (4);

\end{tikzpicture}
}
\hfill
%
\subfigure[]{
\label{sub:bipartite}
\begin{tikzpicture}[every node/.style={default node}]
\node(1) at (0,0) {1};
\node(2) at (2,0) {2};
\node(3) at (0,2) {3};
\node(4) at (2,2) {4};
\node(5) at (0,3.5) {5};

\draw[->, very thick, dashed, blue](1) -- (2);
\draw[->, very thick, dotted, red](4) -- (1);
\draw[->, very thick, dashed, blue](3) -- (4);
\draw[->, very thick, dashed, blue](5) -- (4);

\end{tikzpicture}
}
\hfill
\caption{
\label{fig:spanning-bipartite-graph}
An example of a spanning bipartite graph:
\subref{sub:graph} a directed graph, 
\subref{sub:tree} a possible spanning tree  
\subref{sub:bipartite} a spanning bipartite graph:
$E_{LR}$ is the set of blue, dashed edges, 
and $E_{RL}$ is the red, dotted edge   
}
\end{figure}

For a given spanning bipartite graph $G = (V = (L, R), E)$, 
denote by $E_{LR}$ the edges directed from vertices in $L$ to vertices in $R$,
and by $E_{RL}$ the edges directed from vertices in $R$ to vertices in $L$,
For a set of edges $E$ let $w(E) = \sum_{e \in E}{w(e)}$, 
then, clearly, the following property holds:

\begin{observation}
\label{ob:geq_half}
$ \max\{w(E_{LR}), w(E_{RL})\}  \geq \frac{w(E)}{2} $
\end{observation}


We now describe our algorithm. 
Assume for a moment, 
that every vertex can acts both as a driver and as passenger.
We refer to a matching, under this assumption, as a \emph{super matching}.
We now show how an optimal super matching can be found efficiently.
Given a directed graph $G = (V, E)$,
a weight function $w : E \rightarrow \R$,
and a capacity function $c : V \rightarrow \N$, 
when every vertex can acts both as a driver and as a passenger, 
a \emph{super matching} is a subgraph of $G$, 
where $d_{in}(v) \leq c(v)$, and $d_{out}(v) \leq 1$ for every $v \in V$. 
A maximum super matching can be found by the following reduction 
to a maximum weight flow problem: 
Let $N = (G', s, t, c', w')$ be a flow network, where 
\begin{align*}
G'				& = (P \cup D \cup \{s, t\}, E_{sp} \cup E_{pd} \cup E_{dt})	\\
P				& = \{p_v : v \in V\}					\\
D				& = \{d_v : v \in V\}					\\
E_{sp}			& = \{ (s, p_v) : p_v \in P \}			\\
E_{pd}			& = \{ ((p_u, d_v)) : (u, v) \in E \}	\\
E_{dt}			& = \{ (d_v, t) : d_v \in D \}			\\
c'(s, p_v)		& = c'(p_u, d_v) = 1					\\
c'(d_v, t)		& = c(v)								\\
w'(p_u, d_v)	& = 
\begin{cases}
w(u, v) & \text{if } (p_u, d_v) \in E_{pd} \\
0 & \text{otherwise}	
\end{cases}
\end{align*}

That is, 
we construct a bipartite graph where the left side represent each vertex 
in $V$ being a passenger,
and the right side each vertex in $V$ being a driver.
Figure~\ref{fig:cm-flow} illustrates this flow network.
%
\begin{figure}
\begin{tikzpicture}[
every node/.style={default node},
scale=1,
]

\node(s) at (0,0) {s};

\node[draw=none]() 		at (3,2.7) {$P$};
\node(p0) 				at (3,2) {$p_0$};
\node[draw=none](pdots1)at (3,1) {$\vdots$};
\node(pi) 				at (3,0) {$p_i$};
\node[draw=none](pdots2)at (3,-1) {$\vdots$};
\node(pl) 				at (3,-2) {$p_n$};


\node[draw=none]() 		at (7,2.7) {$D$};
\node(d0) 				at (7,2) {$d_0$};
\node[draw=none](pdots)	at (7,1) {$\vdots$};
\node(dj) 				at (7,0) {$d_j$};
\node[draw=none](pdots)	at (7,-1) {$\vdots$};
\node(dm) 				at (7,-2) {$d_n$};

\node(t) at (10,0) {t};

% s -> P
\draw[->] (s) -- (pi)
node[label above] {$w' = 0$}
node[label below] {$c' = 1$};

% P -> D
\draw[->] (pi) -- (dj)
node[label above, above=-5mm] {$w' = w(i, j)$}
node[label below] {$c' = 1$};

% D -> t
\draw[->] (dj) -- (t)
node[label above] {$w' = 0$}
node[label below, below=-3mm] {$c' = c(j)$};

% dots
\newcommand{\edots}[2]{
\path (#1) -- (#2)
node[label, pos=0.1, anchor=center] {$\cdots$}
node[label, pos=0.9, anchor=center] {$\cdots$};
}

\edots{s}{p0}
\edots{s}{pl}

\edots{p0}{d0}
\edots{p0}{dj}
\edots{pi}{d0}
\edots{pi}{dm}
\edots{pl}{dj}
\edots{pl}{dm}

\edots{d0}{t}
\edots{dm}{t}

\end{tikzpicture}
\caption{
\label{fig:cm-flow}
Illustration of the flow network that is used to find a super-matching.
}
\end{figure}
%
One can verify that this is indeed a flow network and that there is a straight forward 
translation between a flow and a super matching with the same weight.

We now consider the underlying graph of an optimal super matching.
Recall that in a super matching the out degree of every vertex is at most 1,
that is, the underlying graph of a super matching is a set of connected components,
where, the number of edges, 
in every connected component is at most the number of vertices in that component.

We now show how to transform a super matching into a feasible matching.
For every connected component $C = (V, E)$ we construct a feasible matching by
computing a maximum spanning bipartite graph $L = (V, E'_{LR}, E'_{RL})$ of $C$,
and returning $E'_{LR}$ or $E'_{RL}$,
whichever has a higher weight. 
We describe the algorithm in Algorithm~\ref{alg:cm3}.

\begin{algorithm}
\label{alg:cm3}
\KwIn{$G = (V, E), c : V \rightarrow \N, w : E \rightarrow \R$}											 
\KwOut{$(M \subseteq E)$}

$M \leftarrow \emptyset$								\\
$G' \leftarrow \text{superMatching($G$)}$				\\

\For{every connected component $C_i = (V^i, E^i) \in G'$}{	
	$(V^i, E' = (E'_{LR}, E'_{RL})) \leftarrow \text{maximum spanning bipartite graph of } C_i$
	\\
	$M \leftarrow M \cup \argmax_{F \in \{ E'_{LR}, E'_{RL} \}}{w(F)}$	
}

\Return $M$
\caption{Super Matching Algorithm}
\end{algorithm}

We now show that the Super Matching is 3-approximation algorithm.
For the purpose of the analysis, we limit, without loss of generality, 
our discussion to a single connected component.
Let $(V, E)$ be the underlying graph of the super matching, 
and let $M^*$ be an optimal matching, then:
\begin{lemma}
\label{lm:super-geq-m^*}
$w(E) \geq w(M^*)$
\end{lemma}

\begin{proof}
Obviously, every matching is a valid super-matching, 
and $w(E)$ is the weight of an optimal super-matching 
\end{proof}

Let
$$ M = \argmax_{F \in \{ E'_{LR}, E'_{RL} \}}{w(F)} $$
be the matching computed by Algorithm~\ref{alg:cm3}, then:

\begin{lemma}
\label{lm:more_than_e}
$w(M) \geq w(E) - w(E')$
\end{lemma}

\begin{proof}
Recall that $E = E' \cup \{e\}$, 
where $e$ is the lightest edge on some cycle in $(V, E)$. 
Thus, $w(e) = w(E) - w(E')$, 
and there must be another edge $e' \in E \setminus E'$, 
such that $w(e') \geq w(e)$
\end{proof}

\begin{theorem}
Algorithm~\ref{alg:cm3} achieves a 3-approximation ratio.
\end{theorem}

\begin{proof}
From Observation~\ref{ob:geq_half}, Lemma~\ref{lm:super-geq-m^*},
and Lemma~\ref{lm:more_than_e} we get that
$$ 3w(M) = 2w(M) + w(M) \geq w(E') + w(E) - w(E') = w(E) \geq w(M^*)$$.
\end{proof}

\begin{figure}
\centering
\begin{tikzpicture}

\node(n1)[default node] at (0,0) {1};
\node(n2)[default node] at (2,0) {2};
\node(n3)[default node] at (4,0) {3};
\node(n4)[default node] at (2,1.4) {4};

\draw[->, very thick] (n1) -- (n2);
\draw[->, very thick] (n2) -- (n3);
\draw[->, very thick] (n3) to[out=-135, in=-45] (n1);

\foreach \i in {1,...,3}
\draw[->, dashed, very thick] (n\i) -- (n4);

\end{tikzpicture}
\caption{
\label{fig:3cm-tight-fig}
Super Matching Algorithm, worst case example
}
\end{figure}

To see that our analysis is tight, consider the example in Figure~\ref{fig:3cm-tight-fig},
assume, for the given graph in the figure, 
that all weights are 1 and that there is no capacity constraint.
The maximum matching, then, is 3 ($\{(1,4), (2,4), (3,4)\}$), 
but the algorithm can return the super matching $\{(1,2), (2,3), (3,1)\}$ from which, 
only one edge can survive.  