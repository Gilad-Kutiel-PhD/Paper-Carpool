Carpooling, is the sharing of car journeys so that more than one person travels
in a car.
Knapen et al.~\cite{knapen2013estimating} describe an automatic service
to match commuting trips.
Users of the services register their personal profile and a set of periodically
recurring trips, 
and the service advises registered candidates on how to combine their commuting
trips by carpooling.
The services acts in two phases. 

On the first phase, the service estimate the probability that a person $a$
traveling in person's $b$ car will be satisfied from the trip.
This is done based on personal information and feedback from users on past
rides.
The second phase is about finding a carpool matching
that maximizes the global (total expected) satisfaction.

The second phase can be modeled in terms of graph theory.
Given a directed graph $G = (V, E)$.
Each vertex $v \in V$ corresponds to a user of the service and an arc
$(u, v)$ exists if the user corresponded to vertex $u$ is willing to
commute with the user corresponded to vertex $v$.
A capacity function $ c: V \rightarrow \mathbb{N} $ is defined
according to the number of passengers each user can drive if she was
selected as a driver.
A weight function $w : E \rightarrow \R $ defines the amount of
satisfaction $w(u, v)$,
that user $u$ gains when riding with user $v$ for every edge $(u, v) \in E$.

A feasible \emph{carpool matching} (matching) is a triple 
$(P, D, M)$, where $P$ and $D$ form a partition of $V$, and 
$M \subseteq E \cap (P \times D)$,
under the constraints (w.r.t $M$) that for every driver $d \in D$, 
$\din(d) \leq c(d)$, 
and for every passenger $p \in P$, $\dout(p) \leq 1$.
In the \textsc{\CARPOOL{}} problem we seek for a matching $(P, D, M)$ that maximizes the
total weight on the edges.

Here we consider two more variants of the problem, namely, 
the unweighted and the uncapacitated, undirected variants of the problem.
All these variants are NP-hard, to see this, 
consider the unweighted, uncapacitate, undirected variant of the problem, 
the set of vertices with out-degree of zero, in an optimal solution, 
form a minimum dominating set.  
  
Hartman et al.~\cite{hartman2013optimal} proved that the \emph{\CARPOOL{}} problem is
NP-hard, and that it remains NP-hard even for a binary weight function, and even when
the capacity function $c(v) \leq 2$ for every vertex in $V$.
Later, they compared different heuristics algorithms on real data~\cite{hartman2014theory}.
Other heuristics algorithms were developed as well~\cite{knapen2014exploiting}.
When the (potential) set of drivers is known in advanced, however, the problem becomes
tractable and can be solved using a reduction to a flow network.

in Section~\ref{sub:fixed} we discuss how the problem can be solved,
when $P$ and $D$ are fixed,
and in Section~\ref{sub:uwcm} we give a 2-approximation algorithm for the
unweighted variant of the problem,
in Section~\ref{sub:ucudcm} we give a 2-approximation algorithm
for the uncapacitate variants of the problem. 
Finally, in Section~\ref{sub:cm}, we present a 3-approximation
algorithm for the general problem. 
These are the first known approximation algorithms for any variant of the problem.
