Carpooling, is the sharing of car journeys so that more than one person travels
in a car.
Knapen et al.~\cite{knapen2013estimating} describe an automatic service
to match commuting trips.
Users of the services register their personal profile and a set of periodically
recurring trips, 
and the service advises registered candidates on how to combine their commuting
trips by carpooling.
The services acts in two phases. 

On the first phase, the service estimate the probability that a person $a$
traveling in person's $b$ car will be satisfied from the trip.
This is done based on personal information and feedback from users on past
rides.
The second phase is about finding a carpool matching
that maximizes the global (total expected) satisfaction.

The second phase can be modeled in terms of graph theory.
Given a directed graph $G = (V, E)$.
Each vertex $v \in V$ corresponds to a user of the service and an arc
$(u, v)$ exists if the user corresponded to vertex $u$ is willing to
commute with the user corresponded to vertex $v$.
A capacity function $ c: V \rightarrow \N $ is defined
according to the number of passengers each user can drive if she was
selected as a driver.
A weight function $w : E \rightarrow \R $ defines the amount of
satisfaction $w(u, v)$,
that user $u$ gains when riding with user $v$ for every edge $(u, v) \in E$.

A feasible \emph{carpool matching} (matching) is a triple 
$(P, D, M)$, where $P$ and $D$ form a partition of $V$, and 
$M \subseteq E \cap (P \times D)$,
under the constraints that for every driver $d \in D$, 
$\din(d) \leq c(d)$, 
and for every passenger $p \in P$, ${\dout(p) \leq 1}$.
In the \textsc{\CARPOOL{}} problem we seek for a matching $(P, D, M)$ that maximizes the
total weight of $M$.
In other words, the \textsc{\CARPOOL{}} problem is about finding a set of 
(directed toward the center) vertex disjoint stars 
that maximizes the total weights on the edges.
Figure~\ref{fig:carpool} is an example of the \textsc{\CARPOOL{}} problem.
\begin{figure}
\centering
\tikzset{
edge/.style={
->, 
very thick,
},
%
},

\subfloat[]{
\label{sub:input}
\begin{tikzpicture}

\graph[default graph]{
{1/2, 2/3, 3/1, 5/0, 6/4, 8/2, 9/1},
{4/3, 7/3, 10/4},


1 ->[edge label=3] 4,
2 ->[edge label=4] 4,
3 ->[edge label=5] 4,

5 ->[edge label=2] 7,
6 ->[edge label=4] 7,

8 ->[edge label=6] 10,
9 ->[edge label=2] 10,


{1, 4} ->[edge label=4] 5,
{2, 6, 7} ->[edge label=2] 3,
{8, 10} ->[edge label=1] 2,
3 <->[edge label=3] 8,
{1, 4} <->[edge label=1] 9,

};

\end{tikzpicture}
}
\hfill
%
\subfloat[]{
\label{sub:output}
\begin{tikzpicture}

\graph[default graph]{
{[nodes={blue}]1/2, 2/3, 3/1, 5/0, 6/4, 8/2, 9/1},
{[nodes={dashed, red}]4/3, 7/3, 10/4},

1 ->[edge label=3] 4,
2 ->[edge label=4] 4,
3 ->[edge label=5] 4,

5 ->[edge label=2] 7,
6 ->[edge label=4] 7,

8 ->[edge label=6] 10,
9 ->[edge label=2] 10,


{1, 4} ->[draw=none] 5,
{2, 6, 7} ->[draw=none] 3,
{8, 10} ->[draw=none] 2,
3 <->[draw=none] 8,
{1, 4} <->[draw=none] 9,

};
\end{tikzpicture}
}

\caption{
\label{fig:carpool}
A carpool matching example: 
\subref{sub:input} 
a directed graph with capacities on the vertices and weights on the edges 
\subref{sub:output}
a feasible matching with total weight of 26.
$P$ is the set of blue vertices, and $D$ is the set of red, dashed vertices. 
}
\end{figure}  

In this paper, we also consider two more variants of the problem, namely, 
the unweighted and the uncapacitated variants of the problem 
  
Hartman et al.~\cite{hartman2013optimal} proved that all variants of the
\emph{\CARPOOL{}} problem considered in this paper are NP-hard,
and that the problem remains NP-hard even for a binary weight function when
the capacity function $c(v) \leq 2$ for every vertex in $V$.
It is also worth mentioning, that in the undirected
\footnote{The undirected variant of the problem reduces to the directed variant,
by replacing each edge $(u,v)$ by two directed edges,
$(u \rightarrow v)$ and $(v \rightarrow u)$,
each of the same weight as $(u,v)$.}, uncapacitated, unweighted
variant of the problem, the set of drivers, in an optimal solution,
form a minimum dominating set.
When the set of drivers is known in advanced, however, the problem becomes
tractable and can be solved using a reduction to a flow network.

Agatz et al. outlined the optimization challenges that arise 
when developing technology to support ride-sharing and survey the
related operations research models in the academic literature~\cite{agatz2012optimization}.  
Hartman et al. designed several heuristics algorithms for the 
\CARPOOL{} problem and compared 
their performance on real data~\cite{hartman2014theory}.
Other heuristics algorithms were developed as well~\cite{knapen2014exploiting}.
Arkin et al, considered other variants of capacitated star packing where
a capacity vector is given as part of the input and 
capacities are needed to be assigned to vertices~\cite{arkin2004approximations}.  


In Section~\ref{sub:fixed} we discuss how the problem can be solved,
when $P$ and $D$ are fixed.
In Section~\ref{sub:uwcm} we give a 2-approximation algorithm for the
unweighted variant of the problem.
In Section~\ref{sub:ucudcm} we give a 2-approximation algorithm
for the uncapacitated variant of the problem. 
Finally, in Section~\ref{sub:cm}, we present a 3-approximation
algorithm for the general problem. 
These are the first known approximation algorithms for any variant of the problem.
