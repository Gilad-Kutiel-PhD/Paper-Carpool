\label{sub:uwcm}
In this section we present a local search algorithm for the unweighted
variant of the problem.
We show that the approximation ratio of this algorithm is $2$ and give an example
to show that our analysis is tight.

Given a directed graph $G = (V, E)$, 
and a capacity function ${c : V \rightarrow \mathbb{N}}$, 
In the \textsc{\UWCARPOOL{}} problem, 
we seek for a matching that maximize the size of $M$.

We now present a simple local search algorithm for the problem, 
that maintains a feasible matching through its execution.
In every iteration of the algorithm, the size of $M$ increases.
The algorithm terminates, when no further improvement can be done. 

Recall that the \FIXEDCARPOOL{} can be solved efficiently.
Let $M = \text{opt}_{fixed}(P, D)$ be an optimal solution of the
\FIXEDCARPOOL{} problem.
%

For a given matching $M$, denote by $D^M_u \subseteq D$, 
the set of utilized drivers, 
i.e. vertices with an in degree grater than zero.
The local search algorithm, then, tries to improve a given matching, 
by adding another vertex to the set of drivers.
the local search algorithm is described in
Algorithm~\ref{alg:local}.

\begin{algorithm}
\SetKw{True}{true}
\SetKw{False}{false}

\label{alg:local}

\KwIn{$G = (V, E)$, $c : V \rightarrow \mathbb{N}$}
\KwOut{$M$}

$M \leftarrow \emptyset$					\\

\Repeat{done}{
\label{line:outerloop}
	\For{$v \in (V \setminus D^M_u)$}{
		$done$ \leftarrow{} \True			\\
		$D \leftarrow D^M_u \cup \{v\}$ 	\\
		$P \leftarrow V \setminus D$ 		\\
		$M' = \text{opt}_{fixed}(P, D)$ 	\\
		\If{$|M'| > |M|$}{
			$M \leftarrow M'$				\\
			$done$ \leftarrow{} \False		\\
		}
	}
}
\caption{Local Search}
\Return{$M$}
\end{algorithm}

First, observe that the outer loop on line~\ref{line:outerloop} of the local search algorithm
can be executed at most $n$ times, 
where $n$ is the total number of vertices, 
this is because the loop is executed only when there was an improvement, 
and this can happen at most $n$ times.
Also, observe that the body of this loop can be computed in polynomial time, 
and we can conclude that Algorithm~\ref{alg:local} runs in polynomial time.
    
We now prove that the local search algorithm achieves an approximation ratio of
$2$.
Consider a matching as a tuple $(P, D, V_0, M)$, 
where $d_{out}(p) = 1, \, \forall p \in P$,
$d_{in}(d) > 0, \, \forall d \in D$,
and $d_{in} = d_{out} = 0, \, \forall v \in V_0$.
It is easy to see that $P$, $D$, and $V_0$ form a partition of $V$.

Let $(P, D, V_0, M)$ be a matching found by the local search algorithm, and let
$(P^*, D^*, V^*_0, M^*)$ be an arbitrary but fixed optimal matching.
We refer to a driver $v \in D$ with degree $\din(v) = c(v)$ as a \emph{fully utilized}
driver, and denote by $D_f \subseteq D$ the set of fully utilized drivers.

For the purpose of the analysis, assume, without loss of generality, that
if $v \in D_f \cap D^*$, and $(u, v) \in M^*$, then $(u, v) \in M$.
If this is not the case, that is $u \in V_0$, then based on the pigeonhole principle,
there must be a vertex $u' \in P \setminus P^*$, such that $(u', v) \in M \setminus M^*$.
So we can remove $(u', v)$ from $M$ and add $(u, v)$ instead.
Clearly the resulting matching has the same size of the original one,
further, the new matching could actually be found by the $opt_{fixed}$ procedure.


\begin{lemma}
\label{lm:dleqp}
$|D \cap P^*| \leq |P|$
\end{lemma}

\begin{proof}
The lemma follows directly from the observation that 
${|D| \leq |P|}$.
\end{proof}

\begin{lemma}
\label{lm:v_in_p}
If $u \in V_0$ and $(u, v) \in M^*$, then $v \in P \setminus P^*$.
\end{lemma}

\begin{proof}
Recall, that we assume that $v \notin D_f$.
Clearly, if $v$ was in $V_0$ or in $D \setminus D_f$, 
then the solution can be improved by the local search algorithm.
Thus it must be the case that $v \in P$, furthermore, 
if $(u, v) \in M^*$ then $v \notin P^*$. 
\end{proof}

\begin{lemma}
\label{lm:u1nequ2thenv1neqv2}
If $u_1, u_2 \in V_0$, 
$(u_1, v_1), (u_2, v_2) \in M^*$, 
and $u_1 \neq u_2$,
then $v_1 \neq v_2$.
\end{lemma}

\begin{proof}
From Lemma~\ref{lm:v_in_p} we know that $v_1, v_2 \in P$,
Assume for contradiction that $v_1 = v_2 = v$,
and let $w$ be the vertex such that $(v, w) \in M$,
then the local search can improve the solution, 
by replacing $(v, w)$ with $(u_1, v)$ and $(u_2, v)$ - a contradiction.
\end{proof}

\begin{lemma}
\label{lm:unleqp}
$|(V_0 \cup P) \cap P^*| \leq |P|$
\end{lemma}

\begin{proof}
It is enough to show that for every vertex $u \in V_0 \cap P^*$
there is a distinct vertex $v \in P \setminus P^*$.
To see that, for every vertex $u \in V_0 \cap P^*$, 
consider the vertex $v$, such that $(u, v) \in M^*$, 
the proof follows, then, directly
from Lemma~\ref{lm:v_in_p} and Lemma~\ref{lm:u1nequ2thenv1neqv2}.  
\end{proof}


\begin{theorem}
The local search algorithm achieves a 2-approximation ratio.
\end{theorem}

\begin{proof}
By Lemma~\ref{lm:dleqp} and Lemma~\ref{lm:unleqp} we get that 
$$
|P^*| = 
|(P^* \cap D) \cup ((V_0 \cup P) \cap P^*)| \leq 2 \cdot |P|
$$.
\end{proof}

To conclude this section, we show that our analysis is tight.
Consider the example given in Figure~\ref{fig:localtight}.
Assume, in this example, that there are no capacity constraints,
if the local search algorithm start by choosing vertex $3$ to be a driver, 
then the returned matching is the single edge $(2,3)$.
At this point, no further improvement can be done.
The optimal matching, on the other hand, is $\{(1, 2), (3, 2)\}$. 
The path in the example can be duplicated to form an arbitrary large graph (forest).

\begin{figure} 
\centering
\begin{tikzpicture}[every node/.style={default node}]

\node(a) at (0,0) {$1$};
\node(b) at (2,0) {$2$};
\node(c) at (4,0) {$3$};

\draw[->, dashed, very thick] (a) -- (b);
\draw[<->, very thick](b) -- (c);

\end{tikzpicture}

\caption{
\label{fig:localtight}
Local Search - Worst Case Example
}
\end{figure}