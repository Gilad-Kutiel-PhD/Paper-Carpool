\label{sub:uwcm}
In this section we present a local search algorithm for the unweighted
variant of the problem.
We show that the approximation ratio of this algorithm is $2$ and give an example
to show that our analysis is tight.

Given a directed graph $G = (V, E)$, 
and a capacity function ${c : V \rightarrow \N}$, 
In the \textsc{\UWCARPOOL{}} problem, 
we seek for a matching that maximize the size of $M$.

We now present a simple local search algorithm for the problem. 
The algorithm maintains a feasible matching through its execution.
In every iteration of the algorithm, the size of $M$ increases.
The algorithm terminates, when no further improvement can be done. 

Recall that the \FIXEDCARPOOL{} can be solved efficiently.
Let $M = \text{opt}_{fixed}(P, D)$ be an optimal solution of the
\FIXEDCARPOOL{} problem.
%

For a given matching $M$, denote by $D^M_u \subseteq D$, 
the set of utilized drivers, 
i.e. vertices with an in degree grater than zero.
The local search algorithm, then, tries to improve a given matching, 
by adding another vertex to the set of drivers.
the local search algorithm is described in
Algorithm~\ref{alg:local}.

\begin{algorithm}
\SetKw{True}{true}
\SetKw{False}{false}
\KwIn{$G = (V, E)$, $c : V \rightarrow \N$}
\KwOut{$M$}
$M \leftarrow \emptyset$					\\
\Repeat{done}{
\label{line:outerloop}
	$done$ \leftarrow{} \True			\\
	\For{$v \in (V \setminus D^M_u)$}{
		$D \leftarrow D^M_u \cup \{v\}$ 	\\
		$M' = \text{opt}_{fixed}(V \setminus D, D)$ 	\\
		\If{$|M'| > |M|$}{
			$M \leftarrow M'$				\\
			$done$ \leftarrow{} \False		\\
		}
	}
}
\Return{$M$}

\caption{
\label{alg:local}
LocalSearch}
\end{algorithm}

First, observe that the outer loop on line~\ref{line:outerloop} of the local search algorithm
can be executed at most $n$ times, 
where $n$ is the total number of vertices, 
this is because the loop is executed only when there was an improvement, 
and this can happen at most $n$ times.
Also, observe that the body of this loop can be computed in polynomial time, 
and we can conclude that Algorithm~\ref{alg:local} runs in polynomial time.
    
We now prove that the local search algorithm achieves an approximation ratio of $2$.
Consider a matching $(P, D, M)$, and define the following sets:
\begin{itemize}
\item $P_1 = \{p \in P : \dout(p) = 1\}$
\item $D_{>0} = \{d \in D : \din(d) > 0\}$
\item $D_c = \{d \in D_{>0} : \din(d) = c(d)\}$
\item $V_0 = \{v \in P \cup D : \din(v) = \dout(v) = 0\}$ 
\end{itemize}

Let $(P, D, M)$ be a matching found by the local search algorithm, 
and let $P_1$, $D_{>0}$, $D_c$, $V_0$ be the corresponded sets.
Let $(P^*, D^*, M^*)$ be an arbitrary but fixed optimal matching.
and let $P^*_1$ be the corresponded set.
We need to show that $|P^*_1| \leq 2|P_1|$.
To do this we are going to show that 
$|P^*_1 \cap D_{>0}| \leq |P_1|$ 
and that
$|P^*_1 \cap (V_0 \cup P_1)| \leq |P_1|$.

\begin{lemma}
\label{lm:dleqp}
$|P^*_1 \cap D_{>0}| \leq |P_1|$
\end{lemma}

\begin{proof}
The lemma follows directly from the observation that 
${|D_{>0}| \leq |P_1|}$.
\end{proof}

For the rest of the analysis, assume, without loss of generality, that
if $(u, v) \in M^*$ and $u \in V_0$, then $v \notin D_c$.
If this is not the case, that is $v \in D_c$, then based on the pigeonhole principle,
there must be a vertex $u'$, such that $(u', v) \in M \setminus M^*$.
So we can remove $(u', v)$ from $M$ and add $(u, v)$ instead.
Clearly the resulting matching has the same size of the original one,
further, the new matching could actually be found by the $opt_{fixed}$ procedure.
Figure~\ref{fig:uwcm-illustration} depicts the analysis done in Lemma~\ref{lm:v_in_p},
and Lemma~\ref{lm:u1nequ2thenv1neqv2}.

\begin{figure}[ht]
\centering
\begin{tikzpicture}

\node at(0,0) {$P_1$};
\node at(7,3) {$D_{>0}$};
\node at(7,-3) {$V_0$};


\path[draw, pattern=dots, pattern color=red!30]
(4,0) -- (7,0) to[out=90, in=0] (4,3) -- (4,0);

\path[draw, pattern=checkerboard, pattern color=blue!20]
(4,0) -- (7,0) 
to[out=-90, in=0] (4,-3)
to[out=180, in=-90] (1,0)
to[out=90, in=180] (4,3)
-- (4,0);

\draw (5.5,3) circle (0.9) node {$D_c$};

\node[pattern=dots, pattern color=red!20] at(5,1.5) {$P^*_1 \cap D_{>0}$};
\node[pattern=checkerboard, pattern color=blue!15] at(2.9,-1) {$P^*_1 \cap (P_1 \cup V_0)$};

\draw (4,0) -- (8,0);
\draw (4,-4) -- (4,4);

\draw[fill] (6,-1) circle (0.1) node(u) {} node[right] {$u$};
\draw[fill] (6,0.5) circle (0.1) node(v) {} node[right] {$v$};

\draw[fill] (4.5,-1) circle (0.1) node(u1) {} node[above] {$u_1$};
\draw[fill] (5,-2) circle (0.1) node(u2) {} node[below] {$u_2$};
\draw[fill] (2,-3) circle (0.1) node(w) {} node[above] {$w$};

\draw[->, thick] (u) -- (v) node[midway, cross out, draw]{};

\draw[->, thick] (u1) -- (w);
\draw[->, thick] (u2) -- (w) node[midway, cross out, draw]{};

\end{tikzpicture}
\caption{
\label{fig:uwcm-illustration}
Local search analysis:
1) the edge $(a,b)$ implies that $b \in D^*$, thus, $(a,b) \notin M^*$
2) the edges $(c,d)$, and $(e,f)$ do not exist, or otherwise, 
they can be added to $M$ by the local search algorithm
3) we assume that the edge $(g,h) \notin M^*$
4) if the edge $(u_1, v)$ exists, then the edge $(u_2, v)$ does not exists, 
or otherwise, the local search algorithm can increase the size of $M$
}
\end{figure}

\begin{lemma}
\label{lm:v_in_p}
If $(u, v) \in M^*$ and $u \in V_0$, then $v \in P_1 \setminus P^*_1$.
\end{lemma}

\begin{proof}
Recall, that we assume that $v \notin D_c$.
Clearly, if $v$ was in $V_0$ or in $D \setminus D_c$, 
then the solution can be improved by the local search algorithm.
Thus it must be the case that $v \in P$, furthermore, 
if $(u, v) \in M^*$ then $v \notin P^*$. 
\end{proof}

\begin{lemma}
\label{lm:u1nequ2thenv1neqv2}
If $(u_1, v_1), (u_2, v_2) \in M^*$,
$u_1, u_2 \in V_0$, 
and $u_1 \neq u_2$,
then $v_1 \neq v_2$.
\end{lemma}

\begin{proof}
From Lemma~\ref{lm:v_in_p} we know that $v_1, v_2 \in P_1$,
Assume for contradiction that $v_1 = v_2 = v$,
and let $w$ be the vertex such that $(v, w) \in M$,
then the local search can improve the solution, 
by replacing $(v, w)$ with $(u_1, v)$ and $(u_2, v)$ - a contradiction.
\end{proof}

\begin{lemma}
\label{lm:unleqp}
$|P^*_1 \cap (V_0 \cup P_1)| \leq |P_1|$
\end{lemma}

\begin{proof}
It is enough to show that for every vertex $u \in P^*_1 \cap V_0$
there is a distinct vertex $v \in P_1 \setminus P^*_1$.
To see that, for every vertex $u \in P^*_1 \cap V_0$, 
consider the vertex $v$, such that $(u, v) \in M^*$, 
the proof follows, then, directly
from Lemma~\ref{lm:v_in_p} and Lemma~\ref{lm:u1nequ2thenv1neqv2}.  
\end{proof}


\begin{theorem}
\label{th:ls-2}
The local search algorithm achieves a 2-approximation ratio.
\end{theorem}

\begin{proof}
By Lemma~\ref{lm:dleqp} and Lemma~\ref{lm:unleqp} we get that 
$$
|P^*_1| = 
|(P^*_1 \cap D_1) \cup ((V_0 \cup P_1) \cap P^*_1)| \leq 2 \cdot |P_1|
$$.
\end{proof}

To conclude this section, we show that our analysis is tight.
Consider the example given in Figure~\ref{fig:localtight}.
Assume, in this example, that there are no capacity constraints,
if the local search algorithm start by choosing vertex $3$ to be a driver, 
then the returned matching is the single edge $(2,3)$.
At this point, no further improvement can be done.
The optimal matching, on the other hand, is $\{(1, 2), (3, 2)\}$. 
The path in the example can be duplicated to form an arbitrary large graph (forest).

\begin{figure} 
\centering
\begin{tikzpicture}[every node/.style={default node}]

\node(a) at (0,0) {$1$};
\node(b) at (2,0) {$2$};
\node(c) at (4,0) {$3$};

\draw[->, dashed, very thick] (a) -- (b);
\draw[<->, very thick](b) -- (c);

\end{tikzpicture}

\caption{
\label{fig:localtight}
Local Search - Worst Case Example
}
\end{figure}

In Appendix~\ref{apx:k-local} we consider a k-local search algorithm, 
and show that it does not improve over the simple local search algorithm.