\label{sub:uwcm}
In this section we present a local search algorithm for the unweighted
variant of the problem.
We show that the approximation ratio of this algorithm is $2$ and give an example
to show that our analysis is tight.

Given a directed graph $G = (V, E)$, 
and a capacity function ${c : V \rightarrow \N}$, 
In the \textsc{\UWCARPOOL{}} problem, 
we seek for a matching that maximize the size of $M$.

We now present a simple local search algorithm for the problem. 
The algorithm maintains a feasible matching through its execution.
In every iteration of the algorithm, the size of $M$ increases.
The algorithm terminates, when no further improvement can be done. 

Recall that the \FIXEDCARPOOL{} can be solved efficiently.
Let $M = \text{opt}_{fixed}(P, D)$ be an optimal solution of the
\FIXEDCARPOOL{} problem.
%

For a given matching $M$, define the following sets:
\begin{itemize}
\item $P^M = \{p \in P : \dout(p) = 1\}$
\item $D^M = \{d \in D : \din(d) > 0\}$
\item $D^M_c = \{d \in D_{>0} : \din(d) = c(d)\}$
\item $F^M = \{v \in P \cup D : \din(v) = \dout(v) = 0\}$ 
\end{itemize}
We refer to the vertices in these sets as, \emph{passenger}, 
\emph{driver}, \emph{saturated driver}, and \emph{free vertex} respectively.
The local search algorithm, in every iteration, 
tries to improve a the current matching, 
by switching a passenger or a free vertex into a driver 
and compute an optimal fixed matching.
the local search algorithm is described in
Algorithm~\ref{alg:local}.

\begin{algorithm}
\SetKw{True}{true}
\SetKw{False}{false}
\KwIn{$G = (V, E)$, $c : V \rightarrow \N$}
\KwOut{$M$}
$M \leftarrow \emptyset$					\\
\Repeat{done}{
\label{line:outerloop}
	$done$ \leftarrow{} \True				\\
	\For{$v \in (V \setminus D^M)$}{
		$D \leftarrow D^M \cup \{v\}$ 		\\
		$P \leftarrow V \setminus D$ 		\\
		$M' = \text{opt}_{fixed}(P, D)$ 	\\
		\If{$|M'| > |M|$}{
			$M \leftarrow M'$				\\
			$done$ \leftarrow{} \False		\\
		}
	}
}
\Return{$M$}

\caption{
\label{alg:local}
Local Search}
\end{algorithm}

First, observe that the outer loop on line~\ref{line:outerloop} of the local search algorithm
can be executed at most $n$ times, 
where $n$ is the total number of vertices, 
this is because the loop is executed only when there was an improvement, 
and this can happen at most $n$ times.
Also, observe that the body of this loop can be computed in polynomial time, 
and we can conclude that Algorithm~\ref{alg:local} runs in polynomial time.
    
We now prove that the local search algorithm achieves an approximation ratio of $2$.
Let $M$ be a matching found by the local search algorithm, 
and let $M^*$ be an arbitrary but fixed optimal matching.
Observe that every edge in $M^*$ has at least one end point in $M$, formally: 
\begin{lemma}
If $(u,v) \in M^*$, then $\{u,v\} \cap (P^M \cup D^M) \neq \emptyset$
\end{lemma}

\begin{proof}
If this is not the case, Algorithm~\ref{alg:local} can improve $M$
by adding the edge $(u,v)$.  
\end{proof}

Now, with respect to $M$, the optimal solution can not match two free vertices to the same
passenger, formally:
\begin{lemma}
\label{lemma:one-free}
If $(v,w) \in M$, $u_1, u_2 \in F^M$, and $(u_1, v), (u_2, v) \in M^*$, then $u_1 = u_2$.
\end{lemma}

\begin{proof}
If this is not the case, Algorithm~\ref{alg:local} can improve $M$ 
by removing the edge $(v,w)$ and adding the edges $(u_1, v), (u_2, v)$.
\end{proof}

Finally, with respect to $M$, the optimal solution can not match a free vertex to a driver
that is not saturated, formally: 
\begin{lemma}
\label{lemma:saturated}
If $(u,v) \in M^*$, $u \in F^M$, and $v \in D^M$, then $v \in D^M_c$.
\end{lemma}

\begin{proof}
If this is not the case, once again, Algorithm~\ref{alg:local} can improve $M$
by adding the edge $(u,v)$.
\end{proof}

To show that Algorithm~\ref{alg:local} is 2-approximation, 
consider the charging scheme that is illustrated in Figure~\ref{fig:charging}:
Load every edge $(u,v) \in M$ with 2 coins, 
place one coin on $u$ and one coin on $v$.
Observe that every vertex $u \in P^M$ is loaded with one coin, 
and every vertex $v \in D^M$ is loaded with $\din(v)$ coins.   
Now, pay one coin for every $(u,v) \in M^*$, charge $u$ if $u \in P^M \cup D^M$, 
otherwise ($v \in P^M \cup D^M$) charge $v$.
Clearly, every edge in $M^*$ is paid.
We claim that no vertex is overcharged.

\begin{figure}
\centering
\begin{tikzpicture}[very thick, ->]
\tikzset{
	d/.style={red}
}

\graph[default graph]{
	1 -- 2;
};

\end{tikzpicture}
\caption[]{
\label{fig:charging}
Charging Scheme:															\\
1. vertices $1,2,4,5$ are charged with 1\$ each and vertices $3,7$ with 2\$	\\
2. vertex 1 pays for the edge $(1, 4)$										\\
3. vertex 5 pays for the edge $(10, 5)$										\\
4. vertex 7 must be saturated
}
\end{figure}

\begin{lemma}
\label{lemma:p-not-charged}
If $u \in P^M$, then $u$ is not overcharged.
\end{lemma}

\begin{proof}
If $u \in P^{M^*}$, then it is only charged once, otherwise, 
if $u \in D^{M^*}$, then it is only charged for edges $(w, u)$ where $w \in F^M$,
and by lemma~\ref{lemma:one-free}, there is at most one such edge. 
\end{proof}

\begin{lemma}
\label{lemma:d-not-charged}
If $u \in D^M$, then $u$ is not overcharged.
\end{lemma}

\begin{proof}
If $u \in P^{M^*}$, then it is only charged once, 
if $u \in D^{M^*}$, then it is only charged for edges $(w, u)$ where $w \in F^M$,
if such edges exists, then by lemma~\ref{lemma:saturated}, $u$ is saturated, 
and can not be overcharged.
\end{proof}

\begin{theorem}
Algorithm~\ref{alg:local} is 2-approximation
\end{theorem}

\begin{proof}
We used a charging scheme where we managed to pay 1 coin for each edge in $M^*$
by using at most $2|M|$ coins.
\end{proof}


To conclude this section, we show that our analysis is tight.
Consider the example given in Figure~\ref{fig:localtight}.
Assume, in this example, that there are no capacity constraints,
if the local search algorithm start by choosing vertex $3$ to be a driver, 
then the returned matching is the single edge $(2,3)$.
At this point, no further improvement can be done.
The optimal matching, on the other hand, is $\{(1, 2), (3, 2)\}$. 
The path in the example can be duplicated to form an arbitrary large graph (forest).

\begin{figure} 
\centering
\begin{tikzpicture}[every node/.style={default node}]

\node(a) at (0,0) {$1$};
\node(b) at (2,0) {$2$};
\node(c) at (4,0) {$3$};

\draw[->, dashed, very thick] (a) -- (b);
\draw[<->, very thick](b) -- (c);

\end{tikzpicture}

\caption{
\label{fig:localtight}
Local Search - Worst Case Example
}
\end{figure}