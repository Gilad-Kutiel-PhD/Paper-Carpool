\label{sub:bipartition}
Given a directed bipartite graph $G = (L\cup R, E)$, 
let $E_{LR} = E \cap (L \times R)$, 
and let $E_{RL} = E \cap (R \times L)$.
Clearly $E_{LR} \cup E_{RL} = E$, moreover, 
if a weight function $w:E \rightarrow \R$ 
assign weight to every edge in $G$ then the following observation holds:
\begin{observation}
\label{ob:bipartition}
$\max(w(E_{LR}), w(E_{LR})) \geq \frac{w(E)}{2}$
\end{observation}

In particular, given a directed graph $G$, 
and a maximum spanning tree $T$ of $G$, 
applying the original directions of the edges to the edges in $T$ result 
with a directed bipartite graph $\overrightarrow{T}$.
We refer to $\overrightarrow{T}$ as the \emph{maximum spanning bipartite graph}
of $G$. 
Figure~\ref{fig:spanning-bipartite-graph} illustrate this process.   

\begin{figure}
\centering
\subfigure[]{
\label{sub:graph}
\begin{tikzpicture}[every node/.style={default node}]
\node(1) at (0,0) {1};
\node(2) at (2,0) {2};
\node(3) at (0,2) {3};
\node(4) at (2,2) {4};
\node(5) at (1,3.5) {5};

\draw[->](1) -- (2);
\draw[->](1) -- (3);
\draw[->](4) -- (1);
\draw[->](2) -- (3);
\draw[->](2) -- (4);
\draw[->](3) -- (4);
\draw[->](3) -- (5);
\draw[->](5) -- (4);

\end{tikzpicture}
}
\hfill
%
\subfigure[]{
\label{sub:tree}
\begin{tikzpicture}[every node/.style={default node}]
\node(1) at (0,0) {1};
\node(2) at (2,0) {2};
\node(3) at (0,2) {3};
\node(4) at (2,2) {4};
\node(5) at (1,3.5) {5};

\draw(1) -- (2);
\draw(1) -- (4);
\draw(3) -- (4);
\draw(3) -- (5);

\end{tikzpicture}
}
\hfill
%
\subfigure[]{
\label{sub:bipartite}
\begin{tikzpicture}[every node/.style={default node}]
\node(1) at (0,0) {1};
\node(2) at (2,0) {2};
\node(3) at (0,2) {3};
\node(4) at (2,1) {4};
\node(5) at (2,2) {5};

\draw[->, dashed, blue](1) -- (2);
\draw[->, dotted, red](4) -- (1);
\draw[->, dashed, blue](3) -- (4);
\draw[->, dashed, blue](3) -- (5);

\end{tikzpicture}
}
\hfill
\caption{
\label{fig:spanning-bipartite-graph}
An example of a maximum spanning bipartite graph:
\subref{sub:graph} a directed graph, 
\subref{sub:tree} a maximum spanning tree  
\subref{sub:bipartite} a maximum spanning bipartite graph:
$E_{LR}$ is the set of blue edges, 
and $E_{RL}$ is the red, dashed edge   
}
\end{figure}
 