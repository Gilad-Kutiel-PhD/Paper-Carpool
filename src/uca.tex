\section{\uca{}}
\label{sec:uca}
In this section we consider a variant of the \ca{} called \uca{}.
In this variant, there is no capacity constraint on the vertices.
We present an efficient optimal solution for trees,
and a 2-approximation algorithm for general graphs.
We show that our analysis for the approximation algorithm is tight.

\subsection{Optimal Solution for Trees}
Let $T = (V, E)$ be a directed tree.
Let $D \subseteq V$ be the set of drivers, let $P$ be the
set of passengers ($D \ca{}p P = \emptyset$), 
and let $A \subset (E \ca{}p (P \times D))$.
We would like to find $D$, $P$, and $A$ that maximize $w(A)$ such that
$p^A_{out} = 1$ for every $p \in P$.

We define the following functions:
\begin{itemize}
  \item $D(u)$ to be the optimal solution for the sub-tree of $T$ rooted at $u$,
  when $u \in D$.
  \item $P(u)$ to be the optimal solution for the sub-tree of $T$ rooted at
  $u$, when $u \in P$. 
  \item $F(u)$ to be the optimal solution for the sub-tree of $T$ rooted at
  $u$, when $u \notin (D \cup P)$.
\end{itemize}

Then the optimal solution can be calculated using the following
recursion:
\begin{align*}
%
D(u) = \sum_{v : uv \in E}{
\max \begin{cases}
D(v)		\\
P(v)		\\
F(v) + w(vu)
\end{cases}
}
\\
P(u) = \max_{v : uv \in E}{
D(v) + \sum_{w \neq v : uw \in E}{
\max  \begin{cases}
D(w)		\\
P(w)
\end{cases}
}
}
\\
F(u) =  \sum_{v : uv \in E}{
\max \begin{cases}
D(v)		\\
P(v)		
\end{cases}
}
%
\end{align*}

\todo[inline]{proof with induction ?}

\subsection{Approximation Algorithm for General Graphs}
\todo[inline]{currently this section is true for undirected graphs, can we fix
this ?} Our approximation algorithm can be described in two lines:

\begin{algorithm}
\ca{}ption{UCA-APX}
Find a maximal spanning tree of $G$			\;
Return the optimal solution on this tree	\;
\end{algorithm}

\begin{lemma}
The weight of a maximum spanning tree of $G$ is an upper bound on the weight of
an optimal \uca{} in $G$.
\end{lemma}

\begin{proof}
Consider an optimal \uca{}, and complete it to a spanning tree, obviously, the
weight of the resulted tree is no more than the weight of a maximum spanning
tree.
\end{proof}

\begin{theorem}
The UCA-APX algorithm is 2-approximation.
\end{theorem}

\begin{proof}
It is enough to show that for undirected tree, there is always a solution that
weights at least half the weight of the tree.
Let $V_{odd}$, $V_{even}$, $E_{odd}$ and  $E_{even}$
be the set of vertices and edges of the odd and even levels of the tree
respectively.
Then $E_{odd} \cup E_{even} = E$ and $w(E_{odd}) + w(E_{even}) = w(E)$.
\todo[inline]{define a solution to be a triplet (D, P, A)}
Two feasible solutions are $(V_{odd}, V_{even}, E_{odd})$ and 
$(V_{even}, V_{odd}, E_{even})$ with weights $w(E_{odd})$ and $w(E_{even})$,
respectively.
And the proof follow.
\end{proof}
 
 \todo[inline]{give an example for the tight analysis}