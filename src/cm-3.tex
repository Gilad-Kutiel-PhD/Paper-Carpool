\label{sec:cm}
We now present a 3-app\-roximation algorithm for the \textsc{\CARPOOL{}} problem.
This algorithm acts in two phases.
In the first phase it computes a maximum super-matching of $G$, 
in the second phase it turns this super-matching to a feasible matching
by removing some of the edges.

We now show how to transform a super matching into a feasible matching.
First, consider the underlying graph of an optimal super matching.
Recall that in a super matching the out degree of every vertex is at most 1,
that is, the underlying graph of a super matching is a pseudoforest.
For every connected component $C = (V, E)$ 
in the super-matching we construct a feasible matching by
computing a maximum spanning bipartite graph, $T = (V, E_{LR}, E_{RL})$, 
of $C$. 
We then return $E_{LR}$ or $E_{RL}$, whichever weight more. 
We describe the algorithm in Algorithm~\ref{alg:cm3}, 
and illustrate it in Figure~\ref{fig:spanning-bipartite-graph}.   

\begin{algorithm}[t]
\KwIn{$G = (V, E), c : V \rightarrow \N, w : E \rightarrow \R$}											 
\KwOut{$(M \subseteq E)$}

$M \leftarrow \emptyset$								\\
$G' \leftarrow \text{superMatching($G$)}$				\\

\For{every connected component $C_i = (V_i, E_i) \in G'$}{	
	$(V, E_{LR}, E_{RL}) \leftarrow$ maximum spanning bipartite graph of $C_i$	\\
	$M \leftarrow M \cup \argmax_{F \in \{ E_{LR}, E_{RL} \}}{w(F)}$	
}

\Return $M$
\caption{
\label{alg:cm3}
SuperMatching}
\end{algorithm}

\begin{figure}
\centering
\input{fig-super-matching-alg}
\caption[]{
\label{fig:spanning-bipartite-graph}
Illustration of the SuperMatching algorithm:
\subref{subfloat:graph} a directed graph. 
\subref{subfloat:super} a maximum super-matching.  
\subref{subfloat:spanning} a maximum spanning bipartite graph:
$E_{LR}$ is the set of edges exiting red, dashed vertices, 
and $E_{RL}$ is the set of edges exiting blue vertices.
\subref{subfloat:matching} a feasible carpool matching with total value of 6.   
}
\end{figure}

We now show that the SuperMatching algorithm is 3-approximation.
For the purpose of the analysis, we limit, without loss of generality, 
our discussion to a single connected component.
Let $M_{sup}$ be an optimal super matching, 
and let $M^*$ be an optimal matching, then:
\begin{lemma}
\label{lm:super-geq-m^*}
$w(M_{sup}) \geq w(M^*)$
\end{lemma}

\begin{proof}
The proof follows directly from Observation~\ref{ob:matching is super}. 
\end{proof}
%
Let $U = (V, M_{sup})$ be the underlying graph induced by $M_{sup}$, 
and let $\T = (V, T = E_{LR} \cup E_{RL})$ be a maximum spanning bipartite graph of $U$,
and let 
$$M = \argmax_{F \in \{ E_{LR}, E_{RL} \}}{w(F)}$$
be the matching computed by Algorithm~\ref{alg:cm3}, then:

\begin{lemma}
\label{lm:more_than_e}
$w(M) \geq w(M_{sup}) - w(T)$
\end{lemma}

\begin{proof}
Recall that $M_{sup} = T \cup \{e\}$, 
where $e$ is the lightest edge on some cycle in $(V, M_{sup})$. 
Thus, $w(e) = w(M_{sup}) - w(T)$, 
and there must be another edge $e' \in M_{sup} \setminus \{e\}$, 
such that $w(e') \geq w(e)$.
\end{proof}

\begin{theorem}
Algorithm~\ref{alg:cm3} achieves a 3-approximation ratio.
\end{theorem}

\begin{proof}
From Observation~\ref{ob:bipartition}, Lemma~\ref{lm:super-geq-m^*},
and Lemma~\ref{lm:more_than_e} we get that
$$3w(M) = w(M) + 2w(M) \geq w(M_{sup}) - w(T) + w(T) = w(M_{sup}) \geq w(M^*)$$.
\end{proof}

\begin{figure}
\centering
\begin{tikzpicture}

\node(n1)[default node] at (0,0) {1};
\node(n2)[default node] at (2,0) {2};
\node(n3)[default node] at (4,0) {3};
\node(n4)[default node] at (2,2) {4};

\draw[->, very thick] (n1) -- (n2);
\draw[->, very thick] (n2) -- (n3);
\draw[->, very thick] (n3) to[out=-90, in=-90] (n1);

\foreach \i in {1,...,3}
\draw[->, dashed, very thick] (n\i) -- (n4);

\end{tikzpicture}
\caption{
\label{fig:3cm-tight-fig}
Super Matching Algorithm, worst case example
}
\end{figure}

To see that our analysis is tight, consider the example in Figure~\ref{fig:3cm-tight-fig},
assume, for the given graph in the figure, 
that all weights are 1 and that there is no capacity constraint.
The maximum matching, then, is 3 ($\{(1,4), (2,4), (3,4)\}$), 
but the algorithm can return the super matching $\{(1,2), (2,3), (3,1)\}$ from which, 
only one edge can survive.  