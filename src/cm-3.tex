We now present a 3-app\-roximation algorithm for the \textsc{\CARPOOL{}} problem.
This algorithm acts in two phases.
In the first phase it computes a subgraph of $G$ 
where $d_{in}(v) \leq c(v)$, and $d_{out}(v) \leq 1$ for every $v \in V$, 
then, in the second phase it turns this subgraph to a feasible matching
by removing some of the edges.

To begin, assume for a moment, 
that every vertex in the input can acts both as a driver and as passenger.
We refer to a matching, under this assumption, as a \emph{super matching}
We now show how an optimal super matching can be found efficiently.
Given a directed graph $G = (V, E)$,
a weight function $w : E \rightarrow \mathbb{R^+}$,
and a capacity function $c : V \rightarrow \mathbb{N}$, 
when every vertex can acts both as a driver and as a passenger, 
a \emph{super matching} is a subgraph of $G$, 
where $d_{in}(v) \leq c(v)$, and $d_{out}(v) \leq 1$ for every $v \in V$. 
A maximum super matching can be found by the following reduction 
to a maximum weight flow problem: 
Let $N = (G', s, t, c', w')$ be a flow network, where 
\begin{align*}
G' = (V_p \cup V_d \cup \{s, t\}, E_{sp} \cup E_{pd} \cup E_{dt})	\\
V_p = \{v_p : v \in V\} \\
V_d = \{v_d : v \in V\} \\
E_{sp} = \{ (s, u_p) : u_p \in V_p \} \\
E_{pd} = \{ ((u_p, v_d)) : (u, v) \in E \} \\
E_{dt} = \{ (v_d, t) : v_d \in V_d \} \\
c'(u, v) 
\begin{cases}
c(u) & \text{if } (u, v) \in E_{dt} \\
1 & \text{otherwise}	
\end{cases}
\\
w'(u_d, v_p) = 
\begin{cases}
w(u, v) & \text{if } (u_d, v_p) \in E_{pd} \\
0 & \text{otherwise}	
\end{cases}
\end{align*}

That is, 
we construct a bipartite graph where the left side represent each vertex 
in $V$ being a passenger,
and the right side each vertex in $V$ being a driver.
One can verify that this is indeed a flow network and that there is a straight forward 
translation between a flow and a super matching with the same weight.

We now consider the underlying graph of an optimal super matching.
Recall that in a super matching the out degree of every vertex is at most 1,
that is, the underlying graph of a super matching is a set of connected components,
where, the number of edges, 
in every connected component is at most the number of vertices in that component.

We now show how to transform a super matching into a feasible matching.
We can, without loss of generality, 
consider a single connected component in the underline graph of a maximum super matching.
Given a connected component $C = (V, E)$ we construct a feasible matching by
computing a maximum spanning layered graph $L = (V, E'_{even}, E'_{odd})$ of $C$,
and returning the edges in the odd layers or the even layers according to their weights. 

We summarize the algorithm in Algorithm~\ref{alg:cm3}

\begin{algorithm}
\label{alg:cm3}
\KwIn{$G = (V, E), c : V \rightarrow \mathbb{N}, w : E \rightarrow \mathbb{R^+}$}											 
\KwOut{$(M \subseteq E)$}

$E'' \leftarrow \emptyset$								\\
$G' \leftarrow \text{superMatching($G$)}$				\\

\For{every connected component $C_i \in G'$}{	
	$(V^i, E'^i_{even}, E'^i_{odd}) \leftarrow \text{maximum layered spanning graph of } C_i$
	$M \leftarrow E'' \cup \argmax_{E'' \in \{ E'^i_{even}, E'^i_{odd} \}}{w(E'')}$	
}

\Return $E''$
\caption{Super Matching Algorithm}
\end{algorithm}
