\label{sec:ucudcm}
In this section we consider the uncapacitated variant of the problem. 
We give a simple 2-approximation algorithm for this variant, 
and give an example to show that our analysis is tight.

Let $G = (V, E)$ be a directed graph, 
the MaxTree algorithm (Algorithm~\ref{alg:apx})
computes a maximum spanning bipartite graph of $G$, 
$\T = (V, E_{LR} \cup E_{RL})$, 
and returns $E_{LR}$ or $E_{RL}$, 
whichever weights more.

\begin{algorithm}
\caption{
\label{alg:apx}
MaxTree}
$(V, E_{LR}, E_{RL}) \leftarrow$ maximum spanning bipartite graph of $G$	\\
Return $\argmax_{F \in \{ E_{LR}, E_{RL} \}}{w(F)} $						\\
\end{algorithm}

Observed that the MaxTree algorithm returns a feasible matching,
this is because, in the solution, 
there is no vertex that has both incoming and outgoing edges.  
We now prove that the approximation ratio of MaxTree is 2.
We start with a simple lemma.

\begin{lemma}
\label{lm:tree_upper_matching}
The weight of a maximum spanning tree of $G$ is an upper bound on the weight of
an optimal matching in $G$.
\end{lemma}

\begin{proof}
Consider the underlying graph of an optimal matching, 
and complete it to a spanning tree, obviously, the
weight of the resulted tree is no more than the weight of a maximum spanning
tree.
\end{proof}

\begin{theorem}
The MaxTree algorithm achieves approximation ratio of 2.
\end{theorem}

\begin{proof}
The proof follows directly from 
Lemma~\ref{lm:tree_upper_matching} and Observation~\ref{ob:bipartition}.
\end{proof}

To see that the analysis is tight, 
consider a uniform weighted clique with $n$ vertices, 
and a spanning path, 
then the approximation algorithm selects $\frac{n - 1}{2}$ edges, 
while there are $n - 1$ in an optimal solution.  