\label{sec:ucudcm}
In this section we consider the uncapacitated, undirected variant of the problem. 
We give a simple 2-approximation algorithm for this variant, 
and give an example to show that our analysis is tight.

A feasible carpool matching in an undirected graph, $G$, can be found as follow:

\begin{algorithm}
\caption{MaxTree}
Compute a maximum spanning tree, $T$, of $G$.											\\
Pick an arbitrary vertex in $T$ to be its root, and direct the edges toward the root.	\\
\label{item:direct}
Let $(V = L \cup R, E = E_{LR} \cup E_{RL})$ a 2-coloring of $T$						\\
\Return $\argmax_{F \in \{E_{LR}, E_{RL}\}} w(F)$
\end{algorithm}

Observe that step~\ref{item:direct} of the above algorithm ensures that 
the out degree of every vertex is at most 1.
Observe also that in the returned solution
there is no vertex that has both incoming and outgoing edges,
that is the returned solution is a feasible carpool matching.  
We now prove that the approximation ratio of the above algorithm is 2.
We start with a simple observation.

\begin{observation}
\label{lm:tree_upper_matching}
The weight of a maximum spanning tree of $G$ is an upper bound on the weight of
an optimal matching in $G$.
\end{observation}

\begin{proof}
Consider the underlying graph of an optimal matching, 
and complete it to a spanning tree, obviously, the
weight of the resulted tree is no more than the weight of a maximum spanning
tree.
\end{proof}

\begin{theorem}
The MaxTree algorithm achieves approximation ratio of 2.
\end{theorem}

\begin{proof}
The proof follows directly from 
observation~\ref{lm:tree_upper_matching} and Observation~\ref{ob:bipartition}.
\end{proof}

To see that the analysis is tight, 
consider a uniform weighted clique with $n$ vertices, 
and a spanning path, 
then the approximation algorithm selects $\frac{n - 1}{2}$ edges, 
while there are $n - 1$ in an optimal solution.  