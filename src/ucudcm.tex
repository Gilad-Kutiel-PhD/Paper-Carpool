\section{\UCCARPOOL{}}
\label{sub:ucudcm}
In this section we consider a variant of the problem, 
when there are no capacity constraints.
This variant of the problem remains NP-hard even for uniform weights,
this is because, in a maximum matching, the set of drivers form a minimum
dominating set. 
Nevertheless, it turns out, that a very simple algorithm
(Algorithm~\ref{alg:apx}), achieves an approximation ratio of 2.

\begin{algorithm}
\label{alg:apx}
\caption{\UCCARPOOL{}}
$(V, E'_{even}, E'_{odd}) \leftarrow$ maximum spanning layered graph of $G$		\\
Return $\argmax_{E'' \in \{ E'_{even}, E'_{odd} \}}{w(E'')} $	\\
\end{algorithm}

We now prove that the approximation ratio of Algorithm~\ref{alg:apx} is 2.
We start with a simple lemma.

\begin{lemma}
\label{lm:tree_upper_matching}
The weight of a maximum spanning tree of $G$ is an upper bound on the weight of
an optimal matching in $G$.
\end{lemma}

\begin{proof}
Consider an optimal matching, and complete it to a spanning tree, obviously, the
weight of the resulted tree is no more than the weight of a maximum spanning
tree.
\end{proof}

\begin{theorem}
The \UCCARPOOL{} algorithm achieves approximation ratio of 2.
\end{theorem}

\begin{proof}
The proof follows directly from 
Lemma~\ref{lm:tree_upper_matching} and Observation~\ref{ob:geq_half}.
\end{proof}

To see that the analysis is tight, 
consider a uniform weighted clique with $n$ vertices, 
and a spanning path, 
then the approximation algorithm selects $\frac{n - 1}{2}$ edges, 
while there are $n - 1$ in an optimal solution.  