A flow network is a tuple $N = (G = (V, E), s, t, c)$, 
Where $G$ is a directed graph, 
$s \in V$ is a source vertex, 
$t \in V$ is a target vertex, 
and $c : E \rightarrow \R^+$ is a capacity function. 
A flow $f : E \rightarrow \R$ is a function that has the following properties:
\begin{itemize}
\item
$f(e) \leq c(e), \quad \forall e \in E$

\item
$\sum_{(u, v) \in E} f(u, v) = \sum_{(v, w) \in E} f(v, w), \quad \forall v \in V \setminus \{s, t\}$
\end{itemize}

Given a flow function $f$, 
and a weight function $w: E \rightarrow \R$, 
the \emph{flow weight} is defined to be:
$\sum_{e \in E}{w(e)f(e)}$.
A flow with a maximum weight (\emph{maximum weight flow}) can be efficiently found by adding 
the edge $(t, s)$, with $c(t,s) = \infty$, and $w(t,s) = 0$ and reducing the problem
(by switching the sign of the weights) 
to the minimum cost circulation problem~\cite{tardos1985strongly}.

When the capacity function $c$ is integral, 
a maximum weight integral flow can be efficiently found.  
