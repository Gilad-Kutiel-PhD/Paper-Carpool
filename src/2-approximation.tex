\tikzset{m/.style={
	blue, 
	dashed,	
}}

Let $M$ be a feasible carpool matching,
the \emph{value} of a vertex $v$, denoted $val(v)$, 
is the sum of the weights of the edges in $M$ that incident $v$. 
That is: 
%
\begin{definition}[$val$]
$val_M(v) = \sum_{(u, v) \in M} w(u, v) + \sum_{(v, u) \in M} w(v, u)$
\end{definition}
%
If $V$ is a set of vertices, then $val_M(V) \triangleq \sum_{v \in V} val_M(v)$.
%
Denote by $\delta(e)$ the difference between the weight of the edge and the value
of its source vertex,
that is:
%
\begin{definition}[$\delta$]
$\delta_M(u, v) = w(u, v) - val_M(u)$
\end{definition}
%
If $S$ is a set of edges, then $\delta(S) \triangleq \sum_{e \in S}\delta(v)$.

An \emph{improvement} w.r.t vertex $v$ 
is a set of edges entering $v$ of size not greater than the capacity of $v$, 
for which the sum of the $\delta$s is greater than the value of $v$,
formally: 

\begin{definition}[improvement]
$S_v \subseteq V \times \{v\} \cap E$ is an \emph{improvement} (w.r.t $M$)
if the following conditions hold:
\begin{itemize}
\item
$|S_v| \leq c(v)$
\item
$\delta_M(S_v) > val_M(v)$
\end{itemize}  
\end{definition}
%
If, for vertex $v$, an improvement exists, we say that vertex $v$ can be \emph{improved}.
%
Given an edge $(u,v)$, 
let $inc(u,v)$ be the set of edges that incident $(u, v)$,
formally:
%
\begin{definition}[inc]
$inc(u,v) = (V \times \{u, v\} \cup \{u, v\} \times V) \cap E$
\end{definition}
%
If $S$ is a set of edges, then $inc(S) \triangleq \bigcup_{e \in S} inc(e)$.
%
The \emph{passengers} of $v$ (w.r.t some set of edges) are all the vertices that have
outgoing edge to $v$, formally: 
%
\begin{definition}[passengers]
$\text{pass}_M(v) = \{u : (u, v) \in M\}$
\end{definition}
%
Figure~\ref{fig:defs} depicts all the above definitions.

\begin{figure}
\centering
\input{fig-val-delta}
\caption{
\label{fig:defs}
In this example, $M$ is the set of the blue, dashed edges, then:
\begin{itemize}
\item
$val_M(2) = 7$
\item
$val_M(3) = 2$
\item
$val_M(6) = 0$
\item
$\delta_M(2, 3) = 1$
\item
$\delta_M(6, 3) = 2$
\item
the set $\{(2,3), (6,3)\}$ is an \emph{improvement} to vertex 3
\item
$inc(2,3) = \{(1,2),(4,2),(3,5),(6,3)\}$
\item
$\text{pass}_M(2) = \{1, 4\}$
\end{itemize}
}
\end{figure}

We are now ready to describe our algorithm.
The algorithm starts with an empty matching,
in every iteration it seeks for a vertex that can be improved,
it then improve the matching by removing existing edges and adding
the edges that improve the vertex.
The algorithm halts when no vertex can be improved.
We describe the algorithm in Algorithm~\ref{alg:grd}. 

\begin{algorithm}
\caption{
\label{alg:grd}
GRD
}
$M \leftarrow \emptyset$									\\
\Repeat{done}{
	done $\leftarrow$ true									\\
	\For{$v \in V$}{
		\If{$\exists~\text{improvment}~S_v$}{
			$M \leftarrow M \setminus inc(S_v) \cup S_v$	\\
			done $\leftarrow$ \False						\\
		}
	}
}
\end{algorithm}

We argue that determining if a vertex $v$ can be improved can be done efficiently,
by considering the incoming edges to $v$ in a decreasing order of their $\delta$s,
and only ones with positive value.
Vertex $v$ can be improved, then, if the $\delta$s of the first $c(v)$ (or less) edges
sum up to more than $val_M(v)$.

TODO argue that the number of iteration is bounded.    


We now argue that, with respect to any matching, 
the total value of all the vertices equals twice the value of the matching, 
formally: 
\begin{lemma}
\label{lm:val-twice}
$\sum_{v \in V} val_M(v) = 2 \sum_{e \in M} w(e)$
\end{lemma}

\begin{proof}
\begin{equation}
\begin{split}
\sum_{v \in V} val_M(v)	& = 
\sum_{v \in V} \left( \sum_{(u, v) \in M} w(u, v) + \sum_{(v, u) \in M} w(v, u) \right)	\\
						& = \sum_{v \in V}\sum_{(u, v) \in M} w(u, v) + 
							\sum_{v \in V}\sum_{(v, u) \in M} w(v, u)					\\
						& = 2 \sum_{e \in M} w(e)
\end{split}
\end{equation}
\end{proof}

Let $M$ be a matching, and let $v$ be a vertex with no improvement,
let $S \subseteq V \times \{v\} \cap E$ of size not greater than $c(v)$,
then the following lemma holds:

\begin{lemma}
\label{lm:no improve}
$w(S) \leq val_M(v) + val_M(\text{pass}_S(v))$
\end{lemma}

\begin{proof}
If no improvement exists then
$$
w(S) - val_M(\text{pass}_S(v))=
\sum_{(u,v) \in S}(w(u,v) - val_M(u)) =
\delta_M(S) 
\leq val_M(v)
$$
\end{proof}

To bound the approximation ratio of the GRD algorithm, 
we use a charging scheme argument.
Given the matching produced by the algorithm, 
we load every vertex with amount of money equal to its value,
we then show that this is enough to pay for every edge in an optimal matching.   

\begin{theorem}
Algorithm~\ref{alg:grd} is 2-approximation.
\end{theorem}

\begin{proof}
Let $M$ be the matching produced by algorithm~\ref{alg:grd}, 
and let $M^*$ be an optimal matching.
Load every vertex with $val_M(v)$ money, 
by lemma~\ref{lm:val-twice} this is exactly twice the value of $M$.
Consider a driver $v$ in $M^*$, and let $S = V \times \{v\} \cap M^*$.
By lemma~\ref{lm:no improve} we know that $w(S) \leq val_M(v) + val_M(\text{pass}_S(v))$,
thus we can pay for $S$, using the money on $v$ and $\text{pass}_S(v)$.
Clearly, these vertices will not be charged again.
\end{proof}

To conclude this section we show that our analysis is tight, 
as depicts in Figure~\ref{fig:grd worst}.
\begin{figure}
\centering
\input{fig-2-tight}
\caption{
\label{fig:grd worst}
GRD algorithm, worst case example: \\
Consider a path with $2n + 1$ edges, 
and alternating edge weights (2 and 1),
if the GRD algorithm selects all the 1 weighted edges,
then no further improvement can be done and the value of the matching is $n + 1$,
while the optimal matching has value of $2n$. 
}
\end{figure}