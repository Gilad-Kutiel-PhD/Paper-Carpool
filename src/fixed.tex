\label{sec:fixed}
In the \textsc{\FIXEDCARPOOL{}} problem, $P$ and $D$ are given, 
and the goal is to find $M$ that maximizes the total weight. 
This variant of the problem can be solved efficiently
\footnote{A solution to this variant of the problem was already proposed in~\cite{hartman2014theory}.
For the sake of completeness, however, we describe a detailed solution for this variant. 
More importantly, 
the described solution helps us develop the intuition and understand the basic idea behind the
approximation algorithm described in Section~\ref{sec:cm}.   
},
by reducing it to a maximum weight flow (flow) problem as
follow:
Let $(G = (V, E), c, w)$ be a \CARPOOL{} instance,
let $(P, D)$ be a partition of $V$,
let  $N = (G' = (V', E'), s, t, c')$ be a flow network, 
and let $w' : E \rightarrow \N$ be a weight function, where 

\begin{align*}
V'			& = P \cup D \cup \{s, t\}										\\
E'			& = E_{sp} \cup E_{pd} \cup E_{dt}								\\
E_{sp}		& =	\{(s, p) : p \in P \}										\\
E_{pd}		& =	E \cap (P \times D)											\\
E_{dt}		& =	\{(d, t) : d \in D \}										\\
c'(u, v)	& = 
				\begin{cases}
				c(u) & \text{if } (u, v) \in E_{dt} 						\\
				1 & \text{otherwise}
				\end{cases}
																			\\
w'(e)			& = 
				\begin{cases}
				w(e) & \text{if } e \in E_{pd} 								\\
				0 & \text{otherwise}	
				\end{cases}
\end{align*}
The flow network is described in Figure~\ref{fig:flow}
\begin{figure}
\centering
\begin{tikzpicture}[
every node/.style={default node},
scale=1,
]

\node(s) at (0,0) {s};

\node[draw=none]() 		at (3,2.7) {$P$};
\node(p0) 				at (3,2) {$p_0$};
\node[draw=none](pdots1)at (3,1) {$\vdots$};
\node(pi) 				at (3,0) {$p_i$};
\node[draw=none](pdots2)at (3,-1) {$\vdots$};
\node(pl) 				at (3,-2) {$p_l$};


\node[draw=none]() 		at (7,2.7) {$D$};
\node(d0) 				at (7,2) {$d_0$};
\node[draw=none](pdots)	at (7,1) {$\vdots$};
\node(dj) 				at (7,0) {$d_j$};
\node[draw=none](pdots)	at (7,-1) {$\vdots$};
\node(dm) 				at (7,-2) {$d_m$};

\node(t) at (10,0) {t};

% s -> P
\draw[->] (s) -- (pi)
node[label above] {$w' = 0$}
node[label below] {$c' = 1$};

% P -> D
\draw[->] (pi) -- (dj)
node[label above, above=-5mm] {$w' = w(p_i, d_j)$}
node[label below] {$c' = 1$};

% D -> t
\draw[->] (dj) -- (t)
node[label above] {$w' = 0$}
node[label below, below=-3mm] {$c' = c(d_j)$};

% dots
\newcommand{\edots}[2]{
\path (#1) -- (#2)
node[label, pos=0.1, anchor=center] {$\cdots$}
node[label, pos=0.9, anchor=center] {$\cdots$};
}

\edots{s}{p0}
\edots{s}{pl}

\edots{p0}{d0}
\edots{p0}{dj}
\edots{pi}{d0}
\edots{pi}{dm}
\edots{pl}{dj}
\edots{pl}{dm}

\edots{d0}{t}
\edots{dm}{t}

\end{tikzpicture}
\caption{
\label{fig:flow}
Illustration of a flow network corresponded to a \FIXEDCARPOOL{} instance.}
\end{figure}

\begin{observation}
There exists a maximum weight integral flow in $N$.
\end{observation}

\begin{proof}
The observation follows directly from the fact that $c'$ is integral capacity function.
\end{proof}

\begin{observation}
For every integral flow $f$ in $N$, there is a matching $M$ on $G$ with the same weight. 
\end{observation}

\begin{proof}
Consider the matching $(P, D, M^f)$, where 
$$ M^f = \{(p, d) \in E_{pd} : f(p, d) = 1\} $$
one can verify that this is indeed a matching with the same weight as $f$.
\end{proof}

\begin{observation}
For every matching $(P, D, M)$ on $G$, there exists a flow $f$ on $N$ with the same weight. 
\end{observation}

\begin{proof}
Consider the flow function
\begin{align*}
f(s, p_i)		& = \dout(p_i)		 				\\
f(p_i, d_j)		& = 
				\begin{cases}
				1 & \text{if } (p_i, d_j) \in M		\\
				0 & \text{otherwise}
				\end{cases}						\\
f(d_j, t) 	& = \din(d_j) 
\end{align*}

It is easy to verify, that $f$ is indeed a flow function.
Also, observe, that by construction,
the weight of $f$ equals to the weight of the matching.
\end{proof}

As we mentioned in the preliminaries, 
the maximum weight flow problem can be solved efficiently, 
and so is the \FIXEDCARPOOL{} problem.
It is worth mentioning, that it is possible that in a maximum weight flow, 
some of the edges will have no flow at all, 
that is, it is possible that in a \FIXEDCARPOOL{}
some of the passengers and drivers will be unmatched.  